\documentclass[aspectratio=169]{beamer}
\usepackage[utf8]{inputenc}
\usepackage[T1]{fontenc}
\usepackage{graphicx}
\usepackage{booktabs}
\usepackage{xcolor}
\usepackage{tikz}
\usepackage{array}
\usepackage{adjustbox}
\usepackage{hyperref}

% Custom colors
\definecolor{PrimaryBlue}{HTML}{1F4788}
\definecolor{AccentOrange}{HTML}{E87722}
\definecolor{DarkGray}{HTML}{333333}
\definecolor{LightGray}{HTML}{F5F5F5}

% Beamer theme
\usetheme{Madrid}
\usecolortheme[named=PrimaryBlue]{structure}
\setbeamercolor{background canvas}{bg=white}
\setbeamercolor{frametitle}{bg=PrimaryBlue, fg=white}
\setbeamercolor{normal text}{fg=DarkGray}
\setbeamercolor{itemize item}{fg=AccentOrange}

\setbeamerfont{frametitle}{size=\Large, series=\bfseries}
\setbeamerfont{normal text}{size=\normalsize}

\setbeamertemplate{navigation symbols}{}

\title{Staffing Mix Efficiency and Economies of Scale in Nursing Homes}
\author{Elwood Research}
\institute{Center for Health Systems Research}
\date{February 2026}

\begin{document}

% ============================================================================
% SLIDE 1: TITLE
% ============================================================================
\begin{frame}[plain]
  \titlepage
  \vfill
  \begin{center}
    \textit{A quantitative analysis of 14,209 U.S. nursing homes}
    \\[0.3em]
    \textit{CY2023 Q1 -- CY2024 Q4}
  \end{center}
\end{frame}

% ============================================================================
% SLIDE 2: RESEARCH QUESTIONS & BACKGROUND
% ============================================================================
\begin{frame}
  \frametitle{Why Study Facility Size \& Staffing Mix?}
  
  \begin{columns}[T]
    \column{0.55\textwidth}
    \begin{itemize}
      \item \textbf{Staffing Crisis}: RN/LPN shortages in nursing homes are unprecedented
      \item \textbf{Quality Driver}: Care quality directly depends on staffing composition
      \item \textbf{Budget Impact}: Staffing is 60--70\% of operational costs
      \item \textbf{Policy Question}: Do larger facilities achieve better staffing efficiency?
      \item \textbf{Research Gap}: Size-staffing relationship unexplored in PBJ era
    \end{itemize}
    
    \column{0.45\textwidth}
    \centering
    \colorbox{LightGray}{
      \parbox{0.9\linewidth}{
        \vspace{0.3em}
        \centering
        \textbf{Theoretical Basis}
        \vspace{0.3em}
        
        \textit{Transaction Cost Economics predicts that larger organizations achieve economies of scale}
        
        \vspace{0.3em}
      }
    }
  \end{columns}
  
  \vspace{0.8em}
  \centering
  \small \textbf{Central Question:} Does facility size predict more efficient staffing mix?
\end{frame}

% ============================================================================
% SLIDE 3: HYPOTHESES
% ============================================================================
\begin{frame}
  \frametitle{Our Two Hypotheses}
  
  \vspace{0.5em}
  
  \begin{columns}[T]
    \column{0.48\textwidth}
    
    \colorbox{LightGray}{
      \parbox{0.9\linewidth}{
        \vspace{0.3em}
        \centering
        \Large \textbf{H1: RN-to-LPN Ratio}
        \vspace{0.2em}
        
        {\normalsize Larger facilities have higher RN-to-LPN ratios}
        
        \vspace{0.3em}
        \normalsize
        \begin{itemize}
          \setlength{\itemsep}{0.05em}
          \item Economies of scale enable specialization
          \item Prediction: Linear positive relationship
          \item Expected: 50\% increase across facility sizes
        \end{itemize}
        \vspace{0.3em}
      }
    }
    
    \column{0.48\textwidth}
    
    \colorbox{LightGray}{
      \parbox{0.9\linewidth}{
        \vspace{0.3em}
        \centering
        \Large \textbf{H2: Contract CNA}
        \vspace{0.2em}
        
        {\normalsize Mid-sized facilities rely most on contract labor}
        
        \vspace{0.3em}
        \normalsize
        \begin{itemize}
          \setlength{\itemsep}{0.05em}
          \item Mid-size = maximum structural vulnerability
          \item Prediction: Inverted-U relationship
          \item Expected: 8--10\% peak at median size
        \end{itemize}
        \vspace{0.3em}
      }
    }
  \end{columns}
\end{frame}

% ============================================================================
% SLIDE 4: METHODS
% ============================================================================
\begin{frame}
  \frametitle{Cross-Sectional Analysis of 14,209 Nursing Homes}
  
  \begin{itemize}
    \item \textbf{Data Source:} CMS Payroll-Based Journal (PBJ), 8 quarters (CY2023--2024)
    \item \textbf{Sample:} 14,209 Medicare-certified U.S. nursing homes (94.9\% retention after exclusions)
    \item \textbf{Primary Outcomes:}
      \begin{itemize}
        \item RN-to-LPN ratio = RN hours per resident day $\div$ LPN hours per resident day
        \item Contract CNA proportion = Contract CNA hours $\div$ Total CNA hours
      \end{itemize}
    \item \textbf{Main Exposure:} Daily Resident Census (continuous measure of facility size)
    \item \textbf{Statistical Method:} Restricted Cubic Spline (RCS) regression with state fixed effects
    \item \textbf{Robustness Checks:} 6 pre-specified sensitivity analyses (outlier thresholds, minimum census variations)
  \end{itemize}
  
  \vspace{0.8em}
  \textbf{Geographic Coverage:} 52 states and territories; Mean facility size: 79.9 $\pm$ 38.6 residents/day
\end{frame}

% ============================================================================
% SLIDE 5: DESCRIPTIVE FINDINGS
% ============================================================================
\begin{frame}
  \frametitle{Study Sample Characteristics}
  
  \begin{columns}[T]
    \column{0.48\textwidth}
    
    \textbf{Facility Size (Resident Census):}
    \begin{itemize}
      \setlength{\itemsep}{0.2em}
      \item Mean: 79.9 $\pm$ 38.6 residents/day
      \item Median: 75 [IQR: 51--101]
      \item Range: 20--276 residents/day
    \end{itemize}
    
    \vspace{0.7em}
    \textbf{Staffing Hours per Resident Day:}
    \begin{itemize}
      \setlength{\itemsep}{0.2em}
      \item RN: 0.315 $\pm$ 0.229 hrs
      \item LPN: 0.636 $\pm$ 0.387 hrs
      \item CNA (Direct): 1.630 $\pm$ 0.853 hrs
      \item CNA (Contract): 0.115 $\pm$ 0.192 hrs
    \end{itemize}
    
    \column{0.52\textwidth}
    \begin{figure}
      \centering
      \includegraphics[width=\linewidth, height=0.75\textheight, keepaspectratio]{figure1a_census_histogram.png}
      \caption{\small Distribution of Facility Size}
    \end{figure}
  \end{columns}
  
  \vspace{0.2em}
  \small N = 14,209 facilities across 52 states/territories
\end{frame}

% ============================================================================
% SLIDE 6: H1 RESULTS
% ============================================================================
\begin{frame}
  \frametitle{H1 Results: RN-to-LPN Ratio by Facility Size}
  
  \begin{columns}[T]
    \column{0.48\textwidth}
    
    \textbf{HYPOTHESIS CONTRADICTED}
    
    \vspace{0.3em}
    \begin{itemize}
      \setlength{\itemsep}{0.3em}
      \item Larger facilities have LOWER RN-to-LPN ratios (opposite of prediction)
      \item Coefficient: $\beta = -0.00697$ per 10-resident increase
      \item p-value: 0.005
      \item 95\% CI: $[-0.0122, -0.0017]$
      \item Interpretation: Larger facilities employ proportionally MORE LPNs relative to RNs
    \end{itemize}
    
    \vspace{0.8em}
    \textbf{Robustness:} STABLE across all 6 sensitivity analyses
    
    \column{0.52\textwidth}
    \begin{figure}
      \centering
      \includegraphics[width=\linewidth, height=0.75\textheight, keepaspectratio]{figure2_h1_spline_fit.png}
      \caption{\small RN-to-LPN vs. Facility Size. Points = decile means; shaded = 95\% CI}
    \end{figure}
  \end{columns}
\end{frame}

% ============================================================================
% SLIDE 7: H2 RESULTS
% ============================================================================
\begin{frame}
  \frametitle{H2 Results: Contract CNA Proportion by Facility Size}
  
  \begin{columns}[T]
    \column{0.48\textwidth}
    
    \textbf{U-SHAPED PATTERN OBSERVED}
    
    \vspace{0.3em}
    \begin{itemize}
      \setlength{\itemsep}{0.3em}
      \item Small facilities (census $<$50): 6.9\% contract CNA
      \item Mid-sized facilities (census 60--90): 5.4\% contract CNA (minimum)
      \item Large facilities (census $>$150): 6.2\% contract CNA
      \item Linear term: $\beta = -0.000050$ (p = 0.393)
      \item Quadratic term: $\beta = +0.0000003$ (p = 0.286)
      \item Mid-sized facilities achieve lowest contract labor reliance
    \end{itemize}
    
    \vspace{0.8em}
    \textbf{Key Finding:} State/geographic factors explain 11x more variance than facility size
    
    \column{0.52\textwidth}
    \begin{figure}
      \centering
      \includegraphics[width=\linewidth, height=0.75\textheight, keepaspectratio]{figure3_h2_spline_fit.png}
      \caption{\small Contract CNA vs. Facility Size. U-shaped pattern observed}
    \end{figure}
  \end{columns}
\end{frame}

% ============================================================================
% SLIDE 8: GEOGRAPHIC DOMINANCE
% ============================================================================
\begin{frame}
  \frametitle{Geographic Factors Dominate Facility Size}
  
  \textbf{The Big Picture:} State effects explain 11 times more variance than facility size
  
  \vspace{0.6em}
  
  \begin{columns}[T]
    \column{0.5\textwidth}
    
    \textbf{Variance Explained (R²):}
    \begin{itemize}
      \item Facility size alone: 3\%
      \item State fixed effects: 12.2\%
      \item State regulation >> organizational scale
    \end{itemize}
    
    \vspace{0.8em}
    \textbf{Regional Extremes (Contract CNA):}
    \begin{itemize}
      \item Highest: Vermont (+10.9\%)
      \item Maine (+6.7\%), Massachusetts (+5.4\%)
      \item Lowest: Arkansas (-6.7\%)
      \item Range: 17.6 percentage points
    \end{itemize}
    
    \column{0.5\textwidth}
    
    \textbf{State Policy Mechanisms:}
    \begin{itemize}
      \item Medicaid reimbursement rates
      \item State staffing mandates
      \item Licensed labor supply constraints
      \item Union presence and collective bargaining
      \item Post-pandemic workforce dynamics
    \end{itemize}
    
    \vspace{1em}
    \textbf{Implication:} Geographic policy matters far more than facility consolidation
    
  \end{columns}
\end{frame}

% ============================================================================
% SLIDE 9: DISCUSSION & IMPLICATIONS
% ============================================================================
\begin{frame}
  \frametitle{Discussion: Why Did Our Hypotheses Fail?}
  
  \vspace{0.3em}
  
  \textbf{1. Theoretical Implications}
  \begin{itemize}
    \item Transaction Cost Economics assumptions may not hold in regulated healthcare
    \item External constraints (regulation, labor scarcity) override organizational optimization
  \end{itemize}
  
  \vspace{0.5em}
  \textbf{2. Policy Implications}
  \begin{itemize}
    \item CMS minimum staffing standards effectively prevent size-based inequality
    \item State-level Medicaid reimbursement dominates facility-level decisions
  \end{itemize}
  
  \vspace{0.5em}
  \textbf{3. Practice Implications}
  \begin{itemize}
    \item Facility consolidation strategies cannot assume staffing efficiency gains
    \item RN shortage is binding constraint; facility size insufficient to overcome it
    \item Contract labor reliance determined by geography, not organizational scale
  \end{itemize}
  
  \vspace{0.5em}
  \textbf{4. Study Limitations}
  \begin{itemize}
    \item Cross-sectional design: causality not established
    \item No outcome linkage: staffing composition not linked to care quality in this analysis
  \end{itemize}
\end{frame}

% ============================================================================
% SLIDE 10: CONCLUSION & FUTURE DIRECTIONS
% ============================================================================
\begin{frame}
  \frametitle{Conclusion \& Future Directions}
  
  \vspace{0.3em}
  
  \textbf{What We Learned:}
  \begin{itemize}
    \item External regulatory and policy constraints dominate facility-level characteristics
    \item Staffing composition is NOT driven by facility size or economies of scale
    \item Geographic variation (17.6 percentage points) far exceeds facility-size effects
  \end{itemize}
  
  \vspace{0.6em}
  \textbf{Critical Next Steps:}
  \begin{itemize}
    \item Link staffing composition to actual care quality and resident outcomes
    \item Investigate causal mechanisms: regulation or labor markets?
    \item Examine temporal trends: How have staffing patterns evolved post-pandemic?
  \end{itemize}
  
  \vspace{0.6em}
  \textbf{Call to Action:}
  \begin{itemize}
    \item Use findings to inform evidence-based workforce planning
    \item Recognize that consolidation alone cannot solve staffing challenges
    \item Target policy interventions at state/regional level, not facility consolidation
  \end{itemize}
  
  \vspace{0.8em}
  \centering
  \textbf{Research Question Answered:} Facility size does NOT predict staffing efficiency.
\end{frame}

\end{document}
